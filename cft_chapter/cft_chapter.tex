\chapter{Conformal Field Theory}
\label{chap:cft}

\section{What is CFT?}
\label{sec:cft}

I will begin by briefly explaining the relevant aspects of conformal field 
theory. As I mentioned before, conformal field theory is the study of quantum
fields which are broadly stationary under infinitesimal conformal 
transformations; in physics, these transformations can be understood as all 
those with the following action on the metric $g_{\mu \nu}$\cite{Sundrum:2011ic}:

\begin{equation}
    g_{\mu \nu} \to \Omega(x) g_{\mu \nu},
\end{equation}

where $\Omega(x)$ is a function which can depend on the coordinates in any way,
but crucially is applied to every entry of the metric tensor democratically.
Transformations of this form are angle-preserving, in that all vectors which are
orthogonal on the old metric will also be orthogonal on the new metric, and vice
versa. Indeed, the conformal transformations can be thought of as being defined
as the members of the group of angle-preserving maps, which will be more 
familiar to those with a background in mathematics. In fact, in two dimensions 
they are quite the same as the conformal maps which are used extensively in 
complex analysis. In the context of general relativity, being angle-preserving 
means that these transformations also preserve causality, so they can also be
thought of as the most general group of causality-preserving spacetime 
symmetries.

It turns out to be quite straightforward to enumerate the conformal group's 
constituent transformations: they are the Poincar\'e transformations of special
relativity, along with scaling of all coordinates by a constant 
$\vec x \to a \vec x$ (usually referred to as dilatation), and inversion of 
spacetime $\vec x \to \vec x / x^2$; both of these have been centered on 
$\vec 0$, but of course they could be performed about any point. If we want to
preserve orientations in addition to angles, then we have to replace the 
inversion with a \emph{special conformal transformation}, which is an inversion
followed by a translation followed by a second inversion to return to the 
original orientation; these operations are sufficient to reproduce all 
orientation-preserving conformal maps.

If one demands that a theory\footnote{In particle physics parlance, `\textbf{a} 
theory' means a mathematical model for some system (possibly the whole universe)
which includes an inventory of degrees of freedom (generally particles) and a 
prescribed method for calculating their evolution through spacetime. `A 
[relativistic] quantum field theory' means a theory whose degrees of freedom are
quantum fields which are symmetric under Poincar\'e transformations; `a 
conformal field theory' is a QFT which is symmetric under all conformal 
transformations. One can think of `a CFT' as being a particular instantiation of
the more general concept of conformally symmetric models.} respects conformal
symmetry, then the conformal operators ${\cal O}$ representing its quantum 
states must behave in a particular way under infinitesimal conformal 
transformations, such as the infinitesimal dilatation $D$:

\begin{equation}
    \left[ D, {\cal O}_\Delta \right] 
    = (\Delta + x^\mu \partial_\mu) {\cal O}_\Delta.
\end{equation}

$\Delta$, the dilatation eigenvalue, is known as the \emph{[scaling] dimension}
of the operator ${\cal O}_\Delta$, and is an inherent feature of the operator
which is not changed by coordinate transformations. This can be exponentiated to
produce the following behavior under full coordinate scaling:

\begin{equation}
    {\cal O}_\Delta (a x) = a^{-\Delta} {\cal O}_\Delta
\end{equation}

which is the origin of the term \emph{dimension}: in ordinary old-fashioned 
geometry, when the reference length scale is doubled in every direction, the 
size of a 3-dimensional object decreases by a factor of $2^3$, so the scaling 
dimension of a cube is $\Delta = 3$.

\section{Correlation Functions and the OPE}
\label{sec:correlators}

Naturally, there is much, much more which could be said about CFTs, but the
other concept which is central to this paper is the \emph{correlation function}.
This is not originally a CFT concept, but due to the restrictions imposed by
conformal symmetry, it takes a central role in the CFT world: since the two
scalar operators ${\cal O}_\Delta$ and ${\cal O}_{\Delta'}$ have definite
behavior under dilatations and their correlation function must be a scalar due
to Poincar\'e invariance, we can make the following conclusion:
\begin{align}
    \left \langle {\cal O}_\Delta(ax) {\cal O}_{\Delta'}(ax') \right \rangle
    &= \left \langle a^{-\Delta} {\cal O}_\Delta(x) 
                    a^{-\Delta'} {\cal O}_{\Delta'}(x') \right \rangle \\
    &= f((ax - ax')^2) = a^{-\Delta - \Delta'} f((x - x')^2)
    \label{eq:twopoint}
\end{align}
so the correlation function of two conformal operators is another conformal
operator whose scaling dimension is the sum of the two constituent ones. In a 
non-interacting theory, the new operator is the simplest possible one:
\begin{equation}
    f((x - x')^2) = \frac{C(\Delta) \delta_{\Delta \Delta'}}{(x - x')^{2\Delta}}
    \label{eq:twopointfree}
\end{equation}
where $C(\Delta)$ is some constant number; we can get rid of this as well by 
redefining ${\cal O}_\Delta \to {\cal O}_\Delta / \sqrt{C_{\Delta}}$, which is 
always done in practice.

This procedure is closely related to that of the \emph{operator product
expansion}, or OPE, which says that the product of two operators located very
near each other can be represented as a single operator of its own, which can in
turn be represented as an expansion in (i.e. sum over) other operators in the
theory with some constant coefficients $\lambda$:
\begin{equation}
    {\cal O}_A {\cal O}_B = {\cal O}_\mathrm{complicated} 
    = \sum_i \lambda_{A B i} {\cal O}_{i,\mathrm{simple}}.
\end{equation}
Note the lack of brackets: this is not a statement about correlation functions
of operators, but rather one about literally expressing the operators themselves
in a new way; \eqref{eq:twopoint} can be recovered by placing brackets around
the resulting expression, causing everything except the identity 
\eqref{eq:twopointfree} to vanish because conformal symmetry forbids one-point
expectation values for operators.

Once we know that it's possible to replace pairs of operators by their OPE, a
natural next question is ``does this mean that any combination of arbitrarily
many operators can be reduced to a sum over single ones?'' The answer is yes,
and it turns out there's a great deal one can say about these sums. Consider a
4-point correlation function
\begin{equation}
    \left \langle {\cal O}_A {\cal O}_B {\cal O}_C {\cal O}_D \right \rangle
    = \sum_i \sum_j \lambda_{A B i} \lambda_{C D j} \left \langle {\cal O}_i
    {\cal O}_j \right \rangle.
\end{equation}
The $\left \langle {\cal O}_i {\cal O}_j \right \rangle$ above are called
\emph{conformal blocks} and are fixed by conformal symmetry to always be the 
same for a given ${\cal O}_i$ and ${\cal O}_j$, so if a complete set can be
characterized, they can be used to express correlation functions of any set of
four operators (and, in principle, arbitrarily large correlation functions as 
well). In chapter \ref{chap:bh} we numerically characterize the 2-dimensional
conformal blocks and then build four-operator states representing black hole
dynamics with them.

\section{AdS/CFT Holography}
\label{sec:adscft}

In addition to being used to describe systems which are themselves conformally
symmetric (or perhaps merely approximately so), conformal field theory can also
be applied to superficially unrelated problems in quantum gravity. This is due
to the AdS/CFT Correspondence, which says that states in a conformal field
theory are exactly dual to states in a theory of quantum gravity in anti-de 
Sitter spacetime, the maximally symmetric spacetime with a negative cosmological
constant\cite{Maldacena}. The correspondence is such that any dynamic process on 
either side must have a dual process on the other side, which is itself a 
perfectly natural process in its own domain. Crucially, there is no sense in 
which one of these processes is `real' and the other is an `image' of it -- they 
are both valid independently, and neither side can intrinsically be taken to be 
more fundamental than the other. The two sides are descriptions of the same 
thing in two different languages: words, inflection, and even the structure of 
the sentences may be totally different, but they're still talking about the same
object and are capable in principle of expressing all the same ideas about it.

Broadly speaking, research into AdS/CFT tends to focus on taking problems which 
are intractable in either quantum gravity or conformal field theory, translating 
them into the other language, answering them there, and then translating the
results back into the terms of the original problem. This is particularly useful
in quantum gravity, where extremely few problems are readily solved, and about
which very little is known. The next chapter of this thesis concerns one such
project, and we will now turn to its particulars.

