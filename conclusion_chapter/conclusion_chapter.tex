\chapter{Discussion and Conclusion}
\label{chap:conclusion}

The analyses in both of the preceding chapters represent ways of using numerical
computation to attack problems from unexpected angles. Whereas quantum field 
theory is most often solved using perturbation theory in weak interaction, these
projects present entirely orthogonal solutions. In \ref{chap:bh}, the power
series outputted by the program is the \emph{exact} correct answer up to the 
given power in $q$ -- accuracy of the predictions is guaranteed within a given
coordinate range, entirely irrespective of the mass of the objects or the 
strength of the gravitational interaction. In \ref{chap:3d}, the program creates
a Hamiltonian for the system which is exactly correct (up to floating point 
errors) for all of the states included -- the approximation is in ignoring the
higher energy states rather than making any assumptions about the strength of
the interactions.

Accordingly, numerical methods can answer problems which are essentially 
intractable in traditional analytic, proof-based particle physics. The best
known analytic treatment of the AdS$_3$ black hole system remains the 
semiclassical approximation to which we compare our results -- as we 
demonstrated, this approximation completely fails to capture the resolution of
the paradox. The behavior could only be found using numerical methods grounded
in a carefully chosen angle of attack.

Similarly, while the comparisons in \ref{chap:3d} are grounded in the relatively
well understood 3d Ising model, many aspects of this model remain unknown, and
the conformal truncation method we introduce can easily be adapted to study any 
low energy effective theory which can be thought of as a CFT deformed by some 
relevant operator. This is a very broad class of theories, and we expect that
conformal truncation will yield many otherwise inaccessible results when applied
to systems from particle theory and particularly condensed matter.

We would also like to point out that neither of these projects would have been
possible without a cross-disciplinary understanding of various 
efficiency-related topics in computer science. Both make extensive use of 
dynamic programming, and both use a number of data structures and algorithms 
which were carefully selected to minimize repeated work and maximize 
parallelizability without explicit synchronization. We believe that many
outstanding problems in theoretical physics would be amenable to resolution by
a combined approach using both physics and computer science, and would encourage
any physicists interested in computation to improve their understanding of 
algorithm design by working through short CS problems like those at 
\href{https://projecteuler.net}{Project Euler}.

Ultimately, since the problems of theoretical physics are becoming more 
mathematically complex while computers are becoming more powerful, we expect
that the application of cross-disciplinary numerical methods like those used in
this thesis will grow increasingly common. When combined with novel mathematical
approaches like AdS/CFT holography, we believe that computation represents a
compelling forefront for particle physics research, and we are glad to have had
the opportunity to contribute to it.
