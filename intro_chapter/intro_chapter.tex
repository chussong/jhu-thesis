\chapter{Introduction}
\label{chap:intro}

The study of theoretical particle physics has been dominated by relativistic
quantum field theory (QFT) since the middle of the last century
\cite{Weinberg:1995mt}. By far the most well-known outgrowth of QFT is the 
standard model of particle physics, which is by many measures the most precise 
and accurate scientific theory ever created. QFT is extremely versatile, 
however: in addition to the scattering of unimaginably tiny particles inside a 
collider\cite{ref:collider}, it can equally be used to model things as large as 
a condensed matter experiment on a tabletop\cite{ref:condmat} or a black hole 
more massive than the sun\cite{Hawking:1974sw}. Accordingly, if we can detach 
ourselves from models of particular systems and instead focus on results about 
the structure of QFT itself, we can apply that knowledge to very diverse 
scenarios and answer questions from unexpected angles. This is essentially the 
goal of research into pure conformal field theory.

We will go into more detail about the concrete nature of conformal field theory 
in chapter \ref{chap:cft}, but roughly speaking, conformal field theories are 
quantum field theories which behave in qualitatively the same way at any
length scale (or, equivalently, energy scale). The standard model is not a CFT:
most particles have mass energy, so if the total energy of a system is below 
this, it's not possible for that particle to be created, meaning that physics is
qualitatively different below and above this threshold. However, many 
interesting quantum systems \emph{can} be modeled as CFTs, particularly in
condensed matter, where the Ising model near its critical point is the most 
famous example. Additionally, even though particle physics models are generally 
not CFTs overall, they act like CFTs near fixed points of the renormalization
group\cite{Sundrum:2011ic}.

Perhaps even more intriguing is the relatively recently-discovered AdS/CFT
correspondence\cite{Maldacena}. This is a series of mathematical proofs showing 
an exact, one-to-one correspondence between processes in a conformal field 
theory with $d$ spacetime dimensions and quantum gravity processes in anti-de 
Sitter spacetime with $d+1$ dimensions (roughly, the most generic 
$d+1$-dimensional spacetime having a tendency to collapse in on itself). This 
correspondence can be thought of as akin to that between an object in a 
candlelit room and its shadow: when one moves, the other must move as well,
because they are really two views of the same thing.

The AdS/CFT correspondence is much stronger than this, however: looking at a 
shadow, one might easily mistake the head of a duck for a human prankster's 
hand, or vice versa, but quantum operators in the ``shadow'' CFT contain 
\emph{all} of the information in their AdS counterparts, such that it's always 
possible in principle to deduce the full situation in AdS by looking solely at 
the CFT, and vice versa. In fact, the information parity is so complete that 
there is no sense in which one side of the correspondence is more ``real'' than 
the other: they are merely two ways of thinking of the same phenomenon.

The discovery of this correspondence has presented an enormous opportunity for
theoretical physics: any result about a generic CFT can equally be interpreted
through the lens of quantum gravity, and likewise any result from gravity which
holds in a maximally symmetric theory with a negative cosmological constant can 
be translated into a statement about generic conformal field theories. 
Furthermore, many questions with no hint of a clear answer in their own domain 
can be mapped to a dual question on the other side, answered there, and then the 
answer can in turn be mapped back to the original domain. The first project 
presented in this thesis, described in section \ref{sec:bh}, is an example of 
this strategy.

Because generic CFT results can be applied to so many problems in areas as 
diverse as particle physics, quantum gravity, and condensed matter, they are 
quite eagerly sought by physicists of all stripes. The standard way to produce
these results is by careful mathematical proofs. As in pure math, however,
there are many questions with interesting implications for which no solid proof
is forthcoming. In these cases, one can often find sufficiently good approximate
answers by modelling the problem numerically on a computer. This thesis presents
two cases where numerical methods successfully uncovered new insights into 
previously unsolved generic CFT problems.

The first case, described in chapter \ref{chap:bh}, involves the black hole 
information paradox. The behavior of objects falling into black holes is a
quantum gravity process, which under the AdS/CFT correspondence can also be
viewed as a process inside a generic CFT. This process is governed by the
so-called Virasoro Conformal Blocks, a class of functions. The blocks are 
constrained by conformal symmetry to have a particular form, but this form is
known only obliquely through various unsolved equations. Accordingly, if we can
narrow down the form of these functions, we can inspect them for insights into
the gravitational processes at work around black holes. In this project, we used 
a novel algorithm to approximate the Virasoro blocks to a level which was
previously impossible, and thereby extract previously unknown information about
the controversial topic of black hole decay.

The second case, described in chapter \ref{chap:trunc}, concerns the explicit 
calculation of a number of observables in some generic CFTs using the method of
\emph{conformal truncation}. Quantum states in a conformal field theory can be
classified by their eigenvalues under dilatations of spacetime; these 
eigenvalues, $\Delta$, are called the \emph{scaling dimensions} or 
\emph{conformal weights} of the states.

There is a lot to say about scaling dimensions, which will be discussed in a bit
more detail in chapter \ref{chap:cft}, but two properties are particularly 
important for this application: firstly, over relatively long distances, states 
with lower scaling dimensions are more important; secondly, there is a minimum 
scaling dimension, but no maximum (they continue increasing forever). Therefore, 
while there are an infinite number of states in total, we can naturally restrict 
this to a finite number by imposing a cutoff and only considering states below 
that cutoff. As we will discuss, these states with lower dimension will be more
relevant to low energy (long distance) physics, so when that's what we're 
interested in, we should expect to reproduce important behaviors using this 
finite set of states. We were able to show that this is indeed the case, and 
create a program for approximately determining the elements of this basis and 
their interactions with each other. This information can be used to parley 
information about CFTs into information about low energy non-conformal QFT 
phenomena.

% next up: CFT review
